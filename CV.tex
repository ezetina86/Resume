\documentclass[21pt, onecolumn]{article} % Texto a dos columnas
\usepackage[a4paper,left=1.5cm,right=1.5cm]{geometry} % Margenes para el
\usepackage[utf8]{inputenc}
\usepackage[spanish]{babel}
\usepackage{graphicx}
\usepackage{url}
\usepackage{fancyhdr}
\usepackage{pdfpages}
\setlength\parskip{\baselineskip}
\pagestyle{fancy}
\rhead{ \bfseries José Enrique Zetina Moya}
\lfoot{ \bfseries Curriculum Vitae}
\renewcommand{\headrulewidth}{0.3pt}
\renewcommand{\footrulewidth}{0.3pt}

\date{}
\begin{document}

\begin{center}
 \LARGE	 \textbf{\textit{Curriculum Viate}}\\
\end{center}

%\begin{center}
% \includegraphics[scale=.20]{./henry.jpg}\\
%\end{center}

%\section*{Datos Personales}

\begin{tabular}{ l l }

%
%\emph{Nombre:} 		&	José Enrique Zetina Moya\\
%\emph{Dirección:} 	&	Tameses 4 Santa Ursula \\
%					&	Tlalpan CDMX\\
%\emph{Teléfono:} 	&	55 13 15 08 25\\
%\emph{Teléfono celular:} &	55 22 86 69 19\\
%\emph{Correo electrónico:} &	\url{jenzetin@gmail.com}\\
%\emph{Edad:} 		&	31 Años\\
%\emph{Estado Civil:} 	&	Soltero\\

 \end{tabular}

\section*{ Objetivo Profesional }

Resolver de manera creativa y efectiva, problemas que requieran centralizar, depurar y afianzar datos haciendo uso de tecnologías de bases de  datos, actualizarme de forma constante y aprender de nuevas tecnologías con el fin de contribuir al bienestar de mi país y la sociedad en general.

\section*{ Formación académica }

\begin{flushleft}
\begin{itemize}
\begin{flushleft}

 \item Licenciatura en Ingeniería en Computación.\\
  \emph{Universidad Autónoma Metropolitana.}\\
%\emph{Cédula Profesional:} 		89 644 32\\
Proyecto terminal \textbf{``Sistema antivirus y antispam en un
FPGA''}.\\
Estatus:\textit{Titulado. Cédula profesional vigente.}
\end{flushleft}
\end{itemize}
\end{flushleft}

\section*{ Experiencia profesional }

\begin{itemize}
 \item Oracle SQL Sr. Developer\\
  Banco Azteca, Sistemas de Administración de Insumos y Procesos {Noviembre 2017- Marzo 2018}
  \begin{itemize}
  \item Proyecto Control Interno de Expedientes (MER).
     \begin{itemize}
        \item Desarrollo de un sistema permita identificar los elementos generadores de riesgo de cada proceso en sucursales, con el propósito de medir y evaluar el nivel de riesgo operativo integral de las sucursales para poder tomar medidas correctivas y de seguimiento.
        \\En el desarrollo de proyecto contribuí princpalmente desarrollando la lógica de bases de  datos  bajo ORACLE 11G R2, programé módulos de administración y funcionamiento del sistema con el uso de  funciones, procedimientos, vistas, packages, secuencias, etc.
         \\Bajo la metodlogía SCRUM, logré desarrollar  todas las tareas asignadas en tiempo y forma, presentando \emph{sprints} cada dos semanas durante cada fase del proyecto.
        \item Documentación  y manuales de la aplicación.
     \end{itemize}
  \end{itemize}
\end{itemize}

\begin{itemize}
 \item Coodinador Expansión de Negocio\\
  Telefónica Movistar Mexico, {Marzo 2016- Noviembre 2017}
  \begin{itemize}
  \item Segumiento a nuevas aperturas  canal propio y canal especialista:
  \begin{itemize}
  \item Altas.
  \item Ingresos.
  \item Prodocutividad  por $m^2$.
  \end{itemize}
  \item Dieseño de Dashboards para presentciones en alta dirección.
  \item Diseño y desarrollo de soluciones de software en proyectos de business intelligence con uso de ORACLE PL/SQL.
  \item Implementación KPI's para el área de Expansión Comercial.
  \end{itemize}
\end{itemize}

\begin{itemize}
 \item Especialista en Calidad de  Venta\\
 Telefónica Movistar Mexico, {Julio 2013- Marzo 2016}
 \begin{itemize}
%\item Disminución del porcentaje de degradación en el parque recargador por mes de alta.
\item Asegurar  fluidez en los tiempos de respuesta de las solicitudes de información para Prepago a fin de  atender en tiempo y forma las necesidades del área de calidad de la venta.
\item Seguimiento de los requerimientos realizados a sistemas y a las áreas funcionales de la organización para la entrega en tiempo y forma de la información requerida para el monitoreo de la Calidad de Venta.
\item Diseño y desarrollo de soluciones de software en proyectos de business intelligence con uso de ORACLE PL/SQL.
\item ORACLE SQL Tuning.
\item Administración de base de datos ORACLE 11g.
\begin{itemize}
	\item Gestión General de Base de Datos.
	\item Modelado de Datos y Diseño de Base de Datos.
	\item Administración de  usuarios.
	%\item Auditoria.
	%\item Integración con aplicaciones.
	%\item Resguardo y recuperación de datos.
\end{itemize}
\item Implementación de mejoras  y KPI's para el área de Analítica de Negocio.
\item Seguimiento  y control de  canales comerciales y redes de distribución de Telefónica México.
\item Cálculo de penalizaciones en proyectos especiales (Capilaridad,Exportaciones,SIM Movistar).
 \end{itemize}
\end{itemize}

\begin{itemize}
 \item Analista de Cobranza.\\
 Telefónica Movistar Mexico, {2012-2013}
 \begin{itemize}
  \item Segmentación de bases de datos de cartera morosa para su distribución en agencias de
cobranza con el uso de \emph{MySQL}.
 \item Control y seguimiento de facturación electrónica para agencias de
cobranza.
 \item Control y seguimiento de presupuesto del área de cobranza.
%  \item Preparación de devengo para el área de cobranza.
 \end{itemize}

%\begin{itemize}
% \item Profesor.\\
%Instituto de Estudios Avanzados de Enfermería \emph{IEAE}, 2009-2011.
%\begin{itemize}
% \item Enseñanza de habilidades básicas de computación para los estudiantes de
%  enfermería.
%\end{itemize}
\item Supervisor de Operaciones.\\
Atento Servicios, 2008-2009.
\begin{itemize}
\item Supervisor del servicio de Roaming y Servicios Avanzados, para los
clientes de contrato de Movistar México, estando a cargo de un grupo de 20
ejecutivos.
%\end{itemize}
%\item Analista de Datos.\\
%Atento Servicios, 2007-2008.
%\begin{itemize}
% \item Manteniendo diariamente bases de datos de clientes del área de contrato.
% \item Actualizando métricas de productividad y calidad para más de 500
%ejecutivos de atención a clientes en Movistar México.
\end{itemize}
%\item Operador Telefónico.\\
%Atento Servicios, 2007.
%\begin{itemize}
% \item Atención a clientes vía telefónica en la plataforma de Roaming y
%Servicios Avanzados.
% \item Aclaración de los cargos  de las facturas de clientes contrato.
%\end{itemize}
%
%\item Cajero Administrativo \emph{Half Time}.\\
%Grupo Financiero BBVA Bancomer, 2006.
%\begin{itemize}
% \item Atención personal a clientes en sucursal bancaria.
%\end{itemize}

\end{itemize}
%\end{itemize}


\section*{ Habilidades }

\begin{tabular}{ l l l }
Trabajo en equipo  & &  Organización \\
Iniciativa & &  Orientado a resultados \\
Habilidad para la solución de problemas & & Pensamiento crítico   \\
%Inteligencia Emocional
\end{tabular}

\subsection*{ Idiomas }

\begin{itemize}
 \item Ingles  75\%.\\
 Comunicación conversacional y escrita fluida y efectiva.
 \\Dos horas semanales en un club de conversación.
 
\end{itemize}


\subsection*{ Computacionales }

\begin{itemize}
\item Desarrollo de Software bajo metodología \emph{SCRUM \& Agile}.
\item Bases de datos.
\begin{itemize}
 \item Lenguaje de procesamiento procedimental (\emph{PL/SQL})L creación y mantemnimiento de packages, functions, procedures, sequences.
  \item Manejo de lenguajes de definición y manipulación de bases de datos
relacionales (\emph{SQL}).
 \item ETL Performance Tuning: SQL Trace, Explain Plan,Hints, Indexes,
  Partición de  tablas, Subparticiones.
 \item Cargas masivas de información  con SQLloader.
 \item Instalación y configuración de la tecnología GRID(Oracle 11g R2).
% \item Creación y administración de usuarios de  bases de datos.
 \item Instalación y configuración de bases de datos Oracle.
 \item Administración de bases de  datos Oracle.
 %\item Respaldos de bases de  datos Oracle  a través de la  herramienta  RMAN.
 \item Diseño e implementación de bases de datos relacionales. 

 \item Experiencia en la administración de gestores de base de datos como
\emph{Oracle},\emph{SQL Server},\emph{MySQL} y \emph{PostgreSQL}.
  \item Oracle TOAD
  \item SQL Developer
  \item SQLPLUS
 \item Programación en  bases de datos NOSQL \emph{MongoDB}  con el uso de herramientas como :
 \begin{itemize}
  \item Mongo DB Shell
  \item Mongo DB Altas Clustering
  \item Mongo DB Compass
 \end{itemize}
 \item CRUD: Create, Read, Update, and Delete.
 %\item Uso PENTAHO para generar inteligencia de negocios con el uso de herramientas como:
 %\begin{itemize}
 %\item PENTAHO Data Integration
 %\item PENTAHO Report Designer
 %\item PENTAHO Server
% \end{itemize}
\end{itemize}

\item Lenguajes de Programación.
\begin{itemize}
 \item Conocimiento de lenguajes de programación nivel básico \emph{Python} u \emph{Java}.
 \item Diseño ye circuitos electrónicos y sistemas digitales con herramientas de
descripción de hardware \emph{VHDL}.
 \item Desarrollo de sistemas embebidos (\emph{hardware/software}) basados en
FPGAs, utilizando herramientas de Xilinx (ISE/EDK) en Linux.
\end{itemize}

\item Sistemas Operativos.
\begin{itemize}
%  \item Administrador de sistemas operativos basados en
% \emph{UNIX}(\emph{Linux}).
 \item Conocimientos avanzados en administración de servidores
basados en \emph{UNIX} (\emph{Linux}).
\item Manejo del sistema operativo \emph{CISCO IOS} para la administración y
configuración de dispositivos de red(Routers y Switches).
\end{itemize}

\item Paquetería Ofimática.
\begin{itemize}
 \item Diseño de tableros y dashboards con uso de herramientas de inteligencia de negocios, princpalmente \emph{Microsoft BI}.
 \item Manejo de programación de macros para automatización de procesos recurrentes usando VBA.
 \item Amplia experiencia en manejo de hojas de cálculo \emph{Excel}.
 \item Manejo de bases de datos relacionales con la herramienta  \emph{Access}.
 \item Composición de textos orientados especialmente a la creación de
documentos científicos que contengan fórmulas matemáticas con el lenguaje de
composición tipográfica \TeX, especialmente en la forma de {\LaTeX}.
\end{itemize}

\end{itemize}

\section*{Certificaciones}
\begin{itemize}
 \item Oracle Database 11g: Administration Workshop I Release 2.\\
 \emph{Oracle},
Enero 2015. %Oracle Testing ID :  OC1411182 
 \item Oracle Database 11g: SQL Fundamentals I.\\
 \emph{Oracle},
Diciembre 2014. %Oracle Testing ID :  OC1411182 
 \item Certificación Nivel ``A'' Idioma Inglés.\\
 \emph{División de Ciencias Sociales y Humanidades, Lenguas Extranjeras UAM},
Noviembre 2012.
\end{itemize}

\section*{Cursos}
\begin{itemize}
\item M001: MongoDB Basics. Mongo University
\emph{Enero 2018}
\item English Academy Level B2. Dexway
\emph{Septiembre 2017}.
 \item Diplomado Técnico en BIG DATA.
 \emph{Octubre 2017}.
 \item Advanced Conversation Course (ACC).\\
 \emph{Quick Learning} 30 horas, Junio 2012 - Agosto 2012.
 \item TOEFL Course.\\
 \emph{Quick Learning}, con un total de 720 horas de clase, Julio 2010 - Junio
2012.
 \item Curso de Global de Servidores con GNU/Linux.\\
  \emph{Alcance Empresarial}, Diciembre 2011.
 \item Linux Intermedio-Avanzado.\\
  Coordinación de Servicios de Cómputo \emph{UAM}, Octubre 2011.
  \item Curso de Programación de Sistemas.\\
  \emph{Instituto de Ciencias Computacionales}, con un total de 170 horas de
clase, Septiembre 2008.

\end{itemize}

\end{document}


